\PassOptionsToPackage{unicode=true}{hyperref} % options for packages loaded elsewhere
\PassOptionsToPackage{hyphens}{url}
%
\documentclass[]{article}
\usepackage{lmodern}
\usepackage{amssymb,amsmath}
\usepackage{ifxetex,ifluatex}
\usepackage{fixltx2e} % provides \textsubscript
\ifnum 0\ifxetex 1\fi\ifluatex 1\fi=0 % if pdftex
  \usepackage[T1]{fontenc}
  \usepackage[utf8]{inputenc}
  \usepackage{textcomp} % provides euro and other symbols
\else % if luatex or xelatex
  \usepackage{unicode-math}
  \defaultfontfeatures{Ligatures=TeX,Scale=MatchLowercase}
\fi
% use upquote if available, for straight quotes in verbatim environments
\IfFileExists{upquote.sty}{\usepackage{upquote}}{}
% use microtype if available
\IfFileExists{microtype.sty}{%
\usepackage[]{microtype}
\UseMicrotypeSet[protrusion]{basicmath} % disable protrusion for tt fonts
}{}
\IfFileExists{parskip.sty}{%
\usepackage{parskip}
}{% else
\setlength{\parindent}{0pt}
\setlength{\parskip}{6pt plus 2pt minus 1pt}
}
\usepackage{hyperref}
\hypersetup{
            pdftitle={PROJECT},
            pdfauthor={Sukhvir},
            pdfborder={0 0 0},
            breaklinks=true}
\urlstyle{same}  % don't use monospace font for urls
\usepackage[margin=1in]{geometry}
\usepackage{color}
\usepackage{fancyvrb}
\newcommand{\VerbBar}{|}
\newcommand{\VERB}{\Verb[commandchars=\\\{\}]}
\DefineVerbatimEnvironment{Highlighting}{Verbatim}{commandchars=\\\{\}}
% Add ',fontsize=\small' for more characters per line
\usepackage{framed}
\definecolor{shadecolor}{RGB}{248,248,248}
\newenvironment{Shaded}{\begin{snugshade}}{\end{snugshade}}
\newcommand{\AlertTok}[1]{\textcolor[rgb]{0.94,0.16,0.16}{#1}}
\newcommand{\AnnotationTok}[1]{\textcolor[rgb]{0.56,0.35,0.01}{\textbf{\textit{#1}}}}
\newcommand{\AttributeTok}[1]{\textcolor[rgb]{0.77,0.63,0.00}{#1}}
\newcommand{\BaseNTok}[1]{\textcolor[rgb]{0.00,0.00,0.81}{#1}}
\newcommand{\BuiltInTok}[1]{#1}
\newcommand{\CharTok}[1]{\textcolor[rgb]{0.31,0.60,0.02}{#1}}
\newcommand{\CommentTok}[1]{\textcolor[rgb]{0.56,0.35,0.01}{\textit{#1}}}
\newcommand{\CommentVarTok}[1]{\textcolor[rgb]{0.56,0.35,0.01}{\textbf{\textit{#1}}}}
\newcommand{\ConstantTok}[1]{\textcolor[rgb]{0.00,0.00,0.00}{#1}}
\newcommand{\ControlFlowTok}[1]{\textcolor[rgb]{0.13,0.29,0.53}{\textbf{#1}}}
\newcommand{\DataTypeTok}[1]{\textcolor[rgb]{0.13,0.29,0.53}{#1}}
\newcommand{\DecValTok}[1]{\textcolor[rgb]{0.00,0.00,0.81}{#1}}
\newcommand{\DocumentationTok}[1]{\textcolor[rgb]{0.56,0.35,0.01}{\textbf{\textit{#1}}}}
\newcommand{\ErrorTok}[1]{\textcolor[rgb]{0.64,0.00,0.00}{\textbf{#1}}}
\newcommand{\ExtensionTok}[1]{#1}
\newcommand{\FloatTok}[1]{\textcolor[rgb]{0.00,0.00,0.81}{#1}}
\newcommand{\FunctionTok}[1]{\textcolor[rgb]{0.00,0.00,0.00}{#1}}
\newcommand{\ImportTok}[1]{#1}
\newcommand{\InformationTok}[1]{\textcolor[rgb]{0.56,0.35,0.01}{\textbf{\textit{#1}}}}
\newcommand{\KeywordTok}[1]{\textcolor[rgb]{0.13,0.29,0.53}{\textbf{#1}}}
\newcommand{\NormalTok}[1]{#1}
\newcommand{\OperatorTok}[1]{\textcolor[rgb]{0.81,0.36,0.00}{\textbf{#1}}}
\newcommand{\OtherTok}[1]{\textcolor[rgb]{0.56,0.35,0.01}{#1}}
\newcommand{\PreprocessorTok}[1]{\textcolor[rgb]{0.56,0.35,0.01}{\textit{#1}}}
\newcommand{\RegionMarkerTok}[1]{#1}
\newcommand{\SpecialCharTok}[1]{\textcolor[rgb]{0.00,0.00,0.00}{#1}}
\newcommand{\SpecialStringTok}[1]{\textcolor[rgb]{0.31,0.60,0.02}{#1}}
\newcommand{\StringTok}[1]{\textcolor[rgb]{0.31,0.60,0.02}{#1}}
\newcommand{\VariableTok}[1]{\textcolor[rgb]{0.00,0.00,0.00}{#1}}
\newcommand{\VerbatimStringTok}[1]{\textcolor[rgb]{0.31,0.60,0.02}{#1}}
\newcommand{\WarningTok}[1]{\textcolor[rgb]{0.56,0.35,0.01}{\textbf{\textit{#1}}}}
\usepackage{graphicx,grffile}
\makeatletter
\def\maxwidth{\ifdim\Gin@nat@width>\linewidth\linewidth\else\Gin@nat@width\fi}
\def\maxheight{\ifdim\Gin@nat@height>\textheight\textheight\else\Gin@nat@height\fi}
\makeatother
% Scale images if necessary, so that they will not overflow the page
% margins by default, and it is still possible to overwrite the defaults
% using explicit options in \includegraphics[width, height, ...]{}
\setkeys{Gin}{width=\maxwidth,height=\maxheight,keepaspectratio}
\setlength{\emergencystretch}{3em}  % prevent overfull lines
\providecommand{\tightlist}{%
  \setlength{\itemsep}{0pt}\setlength{\parskip}{0pt}}
\setcounter{secnumdepth}{0}
% Redefines (sub)paragraphs to behave more like sections
\ifx\paragraph\undefined\else
\let\oldparagraph\paragraph
\renewcommand{\paragraph}[1]{\oldparagraph{#1}\mbox{}}
\fi
\ifx\subparagraph\undefined\else
\let\oldsubparagraph\subparagraph
\renewcommand{\subparagraph}[1]{\oldsubparagraph{#1}\mbox{}}
\fi

% set default figure placement to htbp
\makeatletter
\def\fps@figure{htbp}
\makeatother


\title{PROJECT}
\author{Sukhvir}
\date{August 20, 2020}

\begin{document}
\maketitle

\hypertarget{r-markdown}{%
\subsection{R Markdown}\label{r-markdown}}

Prediction Assignment Writeup

\hypertarget{analysis}{%
\section{Analysis}\label{analysis}}

Loading data:

\begin{Shaded}
\begin{Highlighting}[]
\NormalTok{training<-}\StringTok{ }\KeywordTok{read.csv}\NormalTok{(}\StringTok{"C:/Users/Sukhvir/Downloads/PML/pml-training.csv"}\NormalTok{)}
\NormalTok{testing<-}\StringTok{ }\KeywordTok{read.csv}\NormalTok{(}\StringTok{"C:/Users/Sukhvir/Downloads/PML/pml-testing.csv"}\NormalTok{)}
\KeywordTok{dim}\NormalTok{(training)}
\end{Highlighting}
\end{Shaded}

\begin{verbatim}
## [1] 19622   160
\end{verbatim}

\begin{Shaded}
\begin{Highlighting}[]
\KeywordTok{dim}\NormalTok{(testing)}
\end{Highlighting}
\end{Shaded}

\begin{verbatim}
## [1]  20 160
\end{verbatim}

\begin{Shaded}
\begin{Highlighting}[]
\CommentTok{# First look at the data}
\KeywordTok{head}\NormalTok{(training)}
\KeywordTok{head}\NormalTok{(testing)}
\end{Highlighting}
\end{Shaded}

\begin{Shaded}
\begin{Highlighting}[]
\CommentTok{# str data}
\KeywordTok{str}\NormalTok{(training)}
\KeywordTok{str}\NormalTok{(testing)}
\end{Highlighting}
\end{Shaded}

\begin{Shaded}
\begin{Highlighting}[]
\CommentTok{# summary}
\KeywordTok{summary}\NormalTok{(training)}
\KeywordTok{summary}\NormalTok{(testing)}
\end{Highlighting}
\end{Shaded}

\hypertarget{cross-validation}{%
\subsection{Cross Validation}\label{cross-validation}}

Cross-validation will be performed by spliting the training dataset
into:

\begin{enumerate}
\def\labelenumi{\arabic{enumi})}
\item
  A training dataset, containing 70\% of the observations. The models
  for prediction will be built using this dataset.
\item
  A testing dataset, containing 30\% of the observations. The accuracy
  of our prediction models will be evaluated using this dataset.
\end{enumerate}

\begin{Shaded}
\begin{Highlighting}[]
\CommentTok{# load packages}
\KeywordTok{library}\NormalTok{(caret)}
\KeywordTok{library}\NormalTok{(randomForest)}
\CommentTok{# Index for training dataset (70%) and testing dataset (30%) }
\CommentTok{# from the pml-training data set}
\KeywordTok{set.seed}\NormalTok{(}\DecValTok{12345}\NormalTok{)}
\NormalTok{inTrain =}\StringTok{ }\KeywordTok{createDataPartition}\NormalTok{(}\DataTypeTok{y=}\NormalTok{training}\OperatorTok{$}\NormalTok{classe,}\DataTypeTok{p=}\FloatTok{0.7}\NormalTok{, }\DataTypeTok{list=}\OtherTok{FALSE}\NormalTok{)}
\CommentTok{# training dataset}
\NormalTok{training.set =}\StringTok{ }\NormalTok{training[inTrain,]}
\CommentTok{# testing dataset}
\NormalTok{testing.set =}\StringTok{ }\NormalTok{training[}\OperatorTok{-}\NormalTok{inTrain,]}
\end{Highlighting}
\end{Shaded}

\hypertarget{training-and-testing}{%
\subsection{Training and Testing}\label{training-and-testing}}

Training and testing data consist of 160 variables. The choice of
specific predictors is based on removing near zero variance predictors,
with the nearZeroVar function, and also variables containing many NAs.

\begin{Shaded}
\begin{Highlighting}[]
\CommentTok{# Remove near zero variance predictors}
\NormalTok{ind.nzv =}\StringTok{ }\KeywordTok{nearZeroVar}\NormalTok{(}\DataTypeTok{x =}\NormalTok{ training, }\DataTypeTok{saveMetrics =}\NormalTok{ T)}
\CommentTok{# Remove variables with more than 50% NA values}
\NormalTok{ind.NA =}\StringTok{ }\OperatorTok{!}\KeywordTok{as.logical}\NormalTok{(}\KeywordTok{apply}\NormalTok{(training, }\DecValTok{2}\NormalTok{, }\ControlFlowTok{function}\NormalTok{(x)\{ }\KeywordTok{mean}\NormalTok{(}\KeywordTok{is.na}\NormalTok{(x)) }\OperatorTok{>=}\StringTok{ }\FloatTok{0.5}\NormalTok{\}))}
\CommentTok{# Cleaning data}
\NormalTok{ind2 =}\StringTok{ }\NormalTok{ind.NA}\OperatorTok{*}\DecValTok{1} \OperatorTok{+}\StringTok{ }\NormalTok{(}\OperatorTok{!}\NormalTok{ind.nzv}\OperatorTok{$}\NormalTok{nzv)}\OperatorTok{*}\DecValTok{1}
\NormalTok{ind3 =}\StringTok{ }\NormalTok{ind2 }\OperatorTok{==}\StringTok{ }\DecValTok{2}
\KeywordTok{sum}\NormalTok{(ind3)}
\end{Highlighting}
\end{Shaded}

\begin{verbatim}
## [1] 59
\end{verbatim}

\begin{Shaded}
\begin{Highlighting}[]
\CommentTok{#View(data.frame(ind.NA, !ind.nzv$nzv, ind2, ind3))}
\NormalTok{training.set =}\StringTok{ }\NormalTok{training.set[,ind3]}
\NormalTok{testing.set =}\StringTok{ }\NormalTok{testing.set[, ind3]}
\NormalTok{training.set =}\StringTok{ }\NormalTok{training.set[, }\DecValTok{-1}\NormalTok{]}
\NormalTok{testing.set =}\StringTok{ }\NormalTok{testing.set[, }\DecValTok{-1}\NormalTok{]}
\NormalTok{testing =}\StringTok{ }\NormalTok{testing[,ind3]}
\NormalTok{testing =}\StringTok{ }\NormalTok{testing[,}\OperatorTok{-}\DecValTok{1}\NormalTok{]}
\CommentTok{# Coerce the data into the same type in order to avoid}
\CommentTok{# "Matching Error" when calling random forest model, due to different levels in variables}
\ControlFlowTok{for}\NormalTok{ (i }\ControlFlowTok{in} \DecValTok{1}\OperatorTok{:}\KeywordTok{length}\NormalTok{(testing) ) \{}
  \ControlFlowTok{for}\NormalTok{(j }\ControlFlowTok{in} \DecValTok{1}\OperatorTok{:}\KeywordTok{length}\NormalTok{(training.set)) \{}
    \ControlFlowTok{if}\NormalTok{( }\KeywordTok{length}\NormalTok{( }\KeywordTok{grep}\NormalTok{(}\KeywordTok{names}\NormalTok{(training.set[i]), }\KeywordTok{names}\NormalTok{(testing)[j]) ) }\OperatorTok{==}\StringTok{ }\DecValTok{1}\NormalTok{)  \{}
      \KeywordTok{class}\NormalTok{(testing[j]) <-}\StringTok{ }\KeywordTok{class}\NormalTok{(training.set[i])}
\NormalTok{    \}      }
\NormalTok{  \}      }
\NormalTok{\}}
\CommentTok{# To get the same class between testing and training.set}
\NormalTok{testing =}\StringTok{ }\NormalTok{testing[,}\OperatorTok{-}\KeywordTok{ncol}\NormalTok{(testing)]}
\NormalTok{testing <-}\StringTok{ }\KeywordTok{rbind}\NormalTok{(training.set[}\DecValTok{2}\NormalTok{, }\DecValTok{-58}\NormalTok{] , testing)}
\NormalTok{testing <-}\StringTok{ }\NormalTok{testing[}\OperatorTok{-}\DecValTok{1}\NormalTok{,]}
\end{Highlighting}
\end{Shaded}

\hypertarget{prediction-model}{%
\subsection{Prediction Model}\label{prediction-model}}

We will use two approaches to create a prediction model for the values
of classe variable.

Firstly prediction with trees will be attempted, using the `rpart'
method and the caret package.

\begin{Shaded}
\begin{Highlighting}[]
\CommentTok{# Prediction with Trees}
\CommentTok{# Build model}
\KeywordTok{set.seed}\NormalTok{(}\DecValTok{12345}\NormalTok{)}
\NormalTok{tree.fit =}\StringTok{ }\KeywordTok{train}\NormalTok{(}\DataTypeTok{y =}\NormalTok{ training.set}\OperatorTok{$}\NormalTok{classe,}
                 \DataTypeTok{x =}\NormalTok{ training.set[,}\OperatorTok{-}\KeywordTok{ncol}\NormalTok{(training.set)],}
                 \DataTypeTok{method =} \StringTok{"rpart"}\NormalTok{)}
\CommentTok{# Plot classification tree}
\NormalTok{rattle}\OperatorTok{::}\KeywordTok{fancyRpartPlot}\NormalTok{(}
\NormalTok{  tree.fit}\OperatorTok{$}\NormalTok{finalModel}
\NormalTok{)}
\end{Highlighting}
\end{Shaded}

\includegraphics{RMD_PROJECT_files/figure-latex/prediction with trees-1.pdf}

\begin{Shaded}
\begin{Highlighting}[]
\CommentTok{# Predictions with rpart model}
\NormalTok{pred.tree =}\StringTok{ }\KeywordTok{predict}\NormalTok{(tree.fit, testing.set[,}\OperatorTok{-}\KeywordTok{ncol}\NormalTok{(testing.set)])}
\CommentTok{# Get results (Accuracy, etc.)}
\KeywordTok{confusionMatrix}\NormalTok{(pred.tree, testing.set}\OperatorTok{$}\NormalTok{classe)}
\end{Highlighting}
\end{Shaded}

\begin{verbatim}
## Confusion Matrix and Statistics
## 
##           Reference
## Prediction    A    B    C    D    E
##          A 1042  108    0    0    0
##          B  632 1023  763  167    0
##          C    0    0    0    0    0
##          D    0    0    0    0    0
##          E    0    8  263  797 1082
## 
## Overall Statistics
##                                           
##                Accuracy : 0.5347          
##                  95% CI : (0.5219, 0.5476)
##     No Information Rate : 0.2845          
##     P-Value [Acc > NIR] : < 2.2e-16       
##                                           
##                   Kappa : 0.4127          
##                                           
##  Mcnemar's Test P-Value : NA              
## 
## Statistics by Class:
## 
##                      Class: A Class: B Class: C Class: D Class: E
## Sensitivity            0.6225   0.8982   0.0000   0.0000   1.0000
## Specificity            0.9744   0.6709   1.0000   1.0000   0.7776
## Pos Pred Value         0.9061   0.3957      NaN      NaN   0.5033
## Neg Pred Value         0.8665   0.9648   0.8257   0.8362   1.0000
## Prevalence             0.2845   0.1935   0.1743   0.1638   0.1839
## Detection Rate         0.1771   0.1738   0.0000   0.0000   0.1839
## Detection Prevalence   0.1954   0.4393   0.0000   0.0000   0.3653
## Balanced Accuracy      0.7984   0.7845   0.5000   0.5000   0.8888
\end{verbatim}

\hypertarget{second-prediction}{%
\subsection{Second Prediction}\label{second-prediction}}

Secondly a prediction model using random forest method will be created.

\begin{Shaded}
\begin{Highlighting}[]
\CommentTok{# Prediction with Random Forest}
\CommentTok{# Build model}
\KeywordTok{set.seed}\NormalTok{(}\DecValTok{12345}\NormalTok{)}
\NormalTok{rf.fit =}\StringTok{ }\KeywordTok{randomForest}\NormalTok{(}
\NormalTok{  classe }\OperatorTok{~}\StringTok{ }\NormalTok{.,}
  \DataTypeTok{data =}\NormalTok{ training.set,}
  \DataTypeTok{ntree =} \DecValTok{250}\NormalTok{)}
\CommentTok{# Plot the Random Forests model}
\KeywordTok{plot}\NormalTok{(rf.fit)}
\end{Highlighting}
\end{Shaded}

\includegraphics{RMD_PROJECT_files/figure-latex/random forest-1.pdf}

\begin{Shaded}
\begin{Highlighting}[]
\CommentTok{# Predict with random forest model}
\NormalTok{pred2 =}\StringTok{ }\KeywordTok{predict}\NormalTok{(}
\NormalTok{  rf.fit,}
\NormalTok{  testing.set[,}\OperatorTok{-}\KeywordTok{ncol}\NormalTok{(testing.set)]}
\NormalTok{)}
\CommentTok{# Get results (Accuracy, etc.)}
\KeywordTok{confusionMatrix}\NormalTok{(pred2, testing.set}\OperatorTok{$}\NormalTok{classe)}
\end{Highlighting}
\end{Shaded}

\begin{verbatim}
## Confusion Matrix and Statistics
## 
##           Reference
## Prediction    A    B    C    D    E
##          A 1674    0    0    0    0
##          B    0 1139    1    0    0
##          C    0    0 1019    5    0
##          D    0    0    6  959    0
##          E    0    0    0    0 1082
## 
## Overall Statistics
##                                           
##                Accuracy : 0.998           
##                  95% CI : (0.9964, 0.9989)
##     No Information Rate : 0.2845          
##     P-Value [Acc > NIR] : < 2.2e-16       
##                                           
##                   Kappa : 0.9974          
##                                           
##  Mcnemar's Test P-Value : NA              
## 
## Statistics by Class:
## 
##                      Class: A Class: B Class: C Class: D Class: E
## Sensitivity            1.0000   1.0000   0.9932   0.9948   1.0000
## Specificity            1.0000   0.9998   0.9990   0.9988   1.0000
## Pos Pred Value         1.0000   0.9991   0.9951   0.9938   1.0000
## Neg Pred Value         1.0000   1.0000   0.9986   0.9990   1.0000
## Prevalence             0.2845   0.1935   0.1743   0.1638   0.1839
## Detection Rate         0.2845   0.1935   0.1732   0.1630   0.1839
## Detection Prevalence   0.2845   0.1937   0.1740   0.1640   0.1839
## Balanced Accuracy      1.0000   0.9999   0.9961   0.9968   1.0000
\end{verbatim}

The accuracy of the random forest model is, as expected, much higher
than the rpart model, over 0.99!

Random Forest model performed better and constitutes the model of choice
for predicting the 20 observations of the original pml-testing.csv
dataset.

\begin{Shaded}
\begin{Highlighting}[]
\CommentTok{# Get predictions for the 20 observations of the original pml-testing.csv}
\NormalTok{pred.validation =}\StringTok{ }\KeywordTok{predict}\NormalTok{(rf.fit, testing)}
\NormalTok{pred.validation}
\end{Highlighting}
\end{Shaded}

\begin{verbatim}
##  1 21  3  4  5  6  7  8  9 10 11 12 13 14 15 16 17 18 19 20 
##  B  A  B  A  A  E  D  B  A  A  B  C  B  A  E  E  A  B  B  B 
## Levels: A B C D E
\end{verbatim}

\begin{Shaded}
\begin{Highlighting}[]
\CommentTok{# Saving predictions for testing dataset}
\NormalTok{testing}\OperatorTok{$}\NormalTok{pred.classe =}\StringTok{ }\NormalTok{pred.validation}
\KeywordTok{write.table}\NormalTok{(}
\NormalTok{  testing,}
  \DataTypeTok{file =} \StringTok{"testing_with_predictions"}\NormalTok{,}
  \DataTypeTok{quote =}\NormalTok{ F}
\NormalTok{)}
\end{Highlighting}
\end{Shaded}

\end{document}
